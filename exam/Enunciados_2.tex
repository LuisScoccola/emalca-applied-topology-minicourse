\documentclass[12pt,reqno,oneside]{article}
\usepackage[spanish]{babel}


\usepackage[letterpaper,hmargin=0.9in,vmargin=0.8in]{geometry}
\usepackage{amsmath}
\usepackage{amsthm}
\usepackage{amssymb}
\usepackage{mathtools}
\usepackage{mathbbol}
\usepackage{fourier}
\usepackage{soul}
\usepackage{tikz}
\usetikzlibrary{matrix,arrows,cd,calc}

\usepackage{accsupp}
\DeclareRobustCommand\squelch[1]{%
    \BeginAccSupp{method=plain,ActualText={}}#1\EndAccSupp{}}

\usepackage[hidelinks]{hyperref}
\usepackage{xcolor}
\definecolor{darkgreen}{rgb}{0,0.45,0}
\hypersetup{colorlinks,urlcolor=blue,citecolor=darkgreen,linkcolor=darkgreen,linktocpage}
\usepackage[capitalize]{cleveref}

%% Notes
\def\noteson{\gdef\note##1{\noindent{\color{blue}[##1]}}}
\gdef\notesoff{\gdef\note##1{\null}}
\noteson
%%

%%% draft watermark
%\usepackage{background}
%\backgroundsetup{
%  position=current page.west,
%  angle=90,
%  nodeanchor=east,
%  vshift=-3mm,
%  opacity=0.1,
%  scale=4,
%  contents=DRAFT,
%  color=black
%}
%%%


%% Custom environments
\newtheorem{theorem}{Theorem}
\newtheorem{lemma}[theorem]{Lemma}
\newtheorem{conjecture}[theorem]{Conjecture}
\newtheorem{proposition}[theorem]{Proposition}
\newtheorem{corollary}[theorem]{Corollary}

\theoremstyle{definition}
\newtheorem{definition}[theorem]{Definition}
\newtheorem{example}[theorem]{Example}
\newtheorem{construction}[theorem]{Construction}
\newtheorem{pregunta}[theorem]{Ejercicio}

%% Custom symbols

\newcommand{\define}[1]{\textbf{\boldmath{#1}}}

\newcommand{\snap}{\mathsf{S}}
\newcommand{\sgn}{\mathsf{sgn}}
\newcommand{\Rect}{\mathsf{Rect}}
\newcommand{\rect}{\mathsf{rect}}
\newcommand{\inproj}{\mathsf{inProj}}
\newcommand{\gldim}{\mathsf{gl.dim}}
\newcommand{\projdim}{\mathsf{proj.dim}}

\renewcommand{\ker}{\mathsf{ker}}
\newcommand{\coker}{\mathsf{coker}}
\newcommand{\im}{\mathsf{im}}

\newcommand{\vect}{\mathsf{vec}}
\newcommand{\persmod}{\mathsf{pmod}}
\newcommand{\grmod}{\mathsf{grmod}}
\newcommand{\modcat}{\mathsf{mod}}

\newcommand{\field}{\mathbb{k}}
\newcommand{\E}{\mathbb{E}}
\newcommand{\R}{\mathbb{R}}
\newcommand{\Z}{\mathbb{Z}}
\newcommand{\N}{\mathbb{N}}
\newcommand{\G}{\mathbb{G}}
\newcommand{\RR}{\mathbf{R}}
\newcommand{\ZZ}{\mathbf{Z}}
\newcommand{\GG}{\mathbf{G}}
\newcommand{\BBB}{\mathcal{B}}
\newcommand{\CCC}{\mathcal{C}}
\newcommand{\EEE}{\mathcal{E}}
\newcommand{\FFF}{\mathcal{F}}
\newcommand{\LLL}{\mathcal{L}}
\newcommand{\TTT}{\mathcal{T}}
\newcommand{\PPP}{\mathcal{P}}
\newcommand{\RRR}{\mathcal{R}}
\newcommand{\SSS}{\mathcal{S}}
\newcommand{\UUU}{\mathcal{U}}
\newcommand{\XXX}{\mathcal{X}}

\renewcommand{\epsilon}{\varepsilon}
\renewcommand{\phi}{\varphi}

\title{Enunciados 2: persistencia para an\'alisis de series de tiempo}
\date{Topolog\'{i}a aplicada -- Emalca 2021 }
\author{Docentes: Jose Perea y Luis Scoccola, Northeastern University}

\begin{document}
\maketitle
%\tableofcontents

\section{Instrucciones generales}
%
\paragraph{C\'odigo.}
Las preguntas est\'an relacionadas al c\'odigo en \texttt{Notebook\_2.ipynb} que puede encontrarse en la p\'agina de la materia
\url{http://luisscoccola.github.io/emalca2021}.

\paragraph{C\'omo responder.}
Por favor responda a las preguntas de la siguiente secci\'on. 
Trate de ser lo m\'as preciso posible y, si no usa \LaTeX~para escribir sus respuestas, escriba lo m\'as claro posible.

\paragraph{C\'omo entregar el examen.}
\textit{Al final del curso}, se deber\'a entregar \textit{un solo archivo} con las respuestas a las preguntas en los Enunciados 1, 2 y 3, que se econtrar\'an en la p\'agina.
La fecha l\'imite para entregar las soluciones es 11:59pm del domingo 21 de noviembre, 2021.
El mail deber\'a tener subject \texttt{examen topologia aplicada -- [SU NOMBRE COMPLETO]}, y deber\'a ser enviado a \squelch{\texttt{luis.scoccola[arroba]gmail.com}}.

\section{Preguntas de An\'alisis de series de tiempo}

Fije $L\in \mathbb{N}$ y sea $f(t) = \sin(Lt)$ para $t\in \mathbb{R}$. 
Recurde que la ventana deslizante de $f$ en $t$, con par\'ametros $d \in \mathbb{N}$ 
y $\tau >0$, est\'a dada por el vector 
\[
SW_{d,\tau} f (t) = 
\begin{bmatrix}
f(t) \\ f(t + \tau) \\  \vdots \\ f(t + d\tau)
\end{bmatrix}
\in \mathbb{R}^{d+1}
\]
El objetivo de este ejercicio es determinar   $d $ y $\tau$ \'optimos, 
de tal forma que la nube de ventanas deslizantes 
\[
\mathbb{SW}_{d,\tau} f = \big\{SW_{d,\tau} f(t) : t\in [0,2\pi]\big\}
\]
tenga persistencia m\'axima en dimensi\'on 1. 
En otras palabras, queremos maximizar
\begin{equation}\label{eq:MaxPers}
mp = \max\Big\{b - a\, :\, (a,b) \in \mathsf{dgm}_1(\mathbb{SW}_{d,\tau} f )\Big\}
\end{equation}

\begin{pregunta}
    Si $f(t) = \sin(Lt)$, muestre que existen vectores $\mathbf{u,v} \in \mathbb{R}^{d+1}$  tales que
    \begin{equation}\label{eq:SW_trig}
    SW_{d,\tau} f(t) = \sin(Lt)\mathbf{u} + \cos(Lt) \mathbf{v}
    \end{equation}
    para todo $t\in \mathbb{R}$. 
    
    \noindent \textbf{Hint:} 
    $\sin(x + y) = \sin(x)\cos(y) + \cos(x)\sin(y)$.
\end{pregunta}

\begin{pregunta}
	Note que si los vectores $\mathbf{u,v}$ son linealmente independientes, entonces la curva  data por la ecuaci\'on (\ref{eq:SW_trig}) es una elipse planar. Cual es el m\'inimo valor que puede tomar $d$ si queremos que $\mathbf{u}$ y $\mathbf{v}$ sean linealmente independientes?
\end{pregunta}
\begin{pregunta}
    Trate de convencerse del siguiente hecho: de todas las elipses planares de per\'imetro fijo, el c\'irculo es el que tiene  1-persistencia m\'axima ($mp$ en ecuaci\'on (\ref{eq:MaxPers})) m\'as grande. 
    Dado este hecho y $d\in \mathbb{N}$ fijo, determine valores de $\tau > 0$
    de tal forma que la ecuaci\'on (\ref{eq:SW_trig}) describa un c\'irculo (redondo).
    
    \noindent \textbf{Hint:} Que valor deber\'ia tener el producto punto de $\mathbf{u}$ y $\mathbf{v}$? Como deber\'ian ser sus longitudes? 
    Las f\'ormulas  para $\cos(2x)$ y $\sin(2y)$, y  las ecuaciones de Lagrange
    \begin{eqnarray*}
    \sum_{j = 0 }^d \sin(2Lj\tau) &=& \frac{\sin(L(d+1)\tau)\sin(Ld\tau)}{\sin(L\tau)} \\
        \sum_{j = 0 }^d \cos(2Lj\tau) &=& \frac{\sin(L(d+1)\tau)\cos(Ld\tau)}{\sin(L\tau)}
    \end{eqnarray*}
    pueden ser \'utiles. 
\end{pregunta}

\end{document}