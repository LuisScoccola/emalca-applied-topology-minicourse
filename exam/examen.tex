\documentclass[12pt,reqno,oneside]{article}
\usepackage[spanish]{babel}


\usepackage[letterpaper,hmargin=0.9in,vmargin=0.8in]{geometry}
\usepackage{amsmath}
\usepackage{amsthm}
\usepackage{amssymb}
\usepackage{mathtools}
\usepackage{mathbbol}
\usepackage{fourier}
\usepackage{soul}
\usepackage{tikz}
\usetikzlibrary{matrix,arrows,cd,calc}

\usepackage[hidelinks]{hyperref}
\usepackage{xcolor}
\definecolor{darkgreen}{rgb}{0,0.45,0}
\hypersetup{colorlinks,urlcolor=blue,citecolor=darkgreen,linkcolor=darkgreen,linktocpage}
\usepackage[capitalize]{cleveref}

%% Notes
\def\noteson{\gdef\note##1{\noindent{\color{blue}[##1]}}}
\gdef\notesoff{\gdef\note##1{\null}}
\noteson
%%

%% draft watermark
\usepackage{background}
\backgroundsetup{
  position=current page.west,
  angle=90,
  nodeanchor=east,
  vshift=-3mm,
  opacity=0.1,
  scale=4,
  contents=DRAFT,
  color=black
}
%%


%% Custom environments
\newtheorem{theorem}{Theorem}
\newtheorem{lemma}[theorem]{Lemma}
\newtheorem{conjecture}[theorem]{Conjecture}
\newtheorem{proposition}[theorem]{Proposition}
\newtheorem{corollary}[theorem]{Corollary}

\theoremstyle{definition}
\newtheorem{definition}[theorem]{Definition}
\newtheorem{example}[theorem]{Example}
\newtheorem{construction}[theorem]{Construction}
\newtheorem{pregunta}[theorem]{Ejercicio}

%% Custom symbols

\newcommand{\define}[1]{\textbf{\boldmath{#1}}}

\newcommand{\snap}{\mathsf{S}}
\newcommand{\sgn}{\mathsf{sgn}}
\newcommand{\Rect}{\mathsf{Rect}}
\newcommand{\rect}{\mathsf{rect}}
\newcommand{\inproj}{\mathsf{inProj}}
\newcommand{\gldim}{\mathsf{gl.dim}}
\newcommand{\projdim}{\mathsf{proj.dim}}

\renewcommand{\ker}{\mathsf{ker}}
\newcommand{\coker}{\mathsf{coker}}
\newcommand{\im}{\mathsf{im}}

\newcommand{\vect}{\mathsf{vec}}
\newcommand{\persmod}{\mathsf{pmod}}
\newcommand{\grmod}{\mathsf{grmod}}
\newcommand{\modcat}{\mathsf{mod}}

\newcommand{\field}{\mathbb{k}}
\newcommand{\E}{\mathbb{E}}
\newcommand{\R}{\mathbb{R}}
\newcommand{\Z}{\mathbb{Z}}
\newcommand{\N}{\mathbb{N}}
\newcommand{\G}{\mathbb{G}}
\newcommand{\RR}{\mathbf{R}}
\newcommand{\ZZ}{\mathbf{Z}}
\newcommand{\GG}{\mathbf{G}}
\newcommand{\BBB}{\mathcal{B}}
\newcommand{\CCC}{\mathcal{C}}
\newcommand{\EEE}{\mathcal{E}}
\newcommand{\FFF}{\mathcal{F}}
\newcommand{\LLL}{\mathcal{L}}
\newcommand{\TTT}{\mathcal{T}}
\newcommand{\PPP}{\mathcal{P}}
\newcommand{\RRR}{\mathcal{R}}
\newcommand{\SSS}{\mathcal{S}}
\newcommand{\UUU}{\mathcal{U}}
\newcommand{\XXX}{\mathcal{X}}

\renewcommand{\epsilon}{\varepsilon}
\renewcommand{\phi}{\varphi}

\title{Examen: Topolog\'{i}a aplicada -- Emalca Per\'{u} 2021}
\date{\today}
\author{Docentes: Jose Perea y Luis Scoccola, Northeastern University}

\begin{document}
\maketitle
%\tableofcontents

\section{Instrucciones}

\paragraph{C\'odigo.}
En \url{https://github.com/LuisScoccola/emalca-applied-topology-minicourse.git}, en el directorio \texttt{notebooks}, encontrar\'a tres Jupyter Notebooks:
\begin{itemize}
	\item \texttt{N1-clustering.ipynb};
	\item \texttt{N2-periodicityDetection.ipynb};
	\item \texttt{N3-poseClassification.ipynb}.
\end{itemize}
Recomendamos ejecutarlas usando \texttt{Google Colab} (\url{https://colab.research.google.com/}).

\paragraph{C\'omo responder.}
Por favor responda a las preguntas en las siguientes tres secciones.
Note que hay \note{TODO} preguntas.
Trate de ser lo m\'as preciso posible y, si no usa \LaTeX~para escribir sus respuestas, escriba lo m\'as claro posible.

\paragraph{C\'omo entregar el examen.}
Mande \textit{un solo archivo} con todas sus respuestas a \note{TODO}.
El mail deber\'a tener subject \texttt{examen topologia aplicada -- [SU NOMBRE Y APELLIDO]}.

\section{Clustering}
\begin{pregunta}
	Que valores de \texttt{KERNEL\_BANDWIDTH}, \texttt{RIPS\_SCALE}, y \texttt{PERSISTENCE\_THRESHOLD} us\'o para el primer dataset?
	Logr\'o encontrar un clustering adecuado?
\end{pregunta}

\begin{pregunta}
	Que valores de \texttt{KERNEL\_BANDWIDTH}, \texttt{RIPS\_SCALE}, y \texttt{PERSISTENCE\_THRESHOLD} us\'o para el segundo dataset?
	Logr\'o encontrar un clustering adecuado?
\end{pregunta}

\begin{pregunta}
	Que valores de \texttt{KERNEL\_BANDWIDTH}, \texttt{RIPS\_SCALE}, y \texttt{PERSISTENCE\_THRESHOLD} us\'o para el tercer dataset?
	Logr\'o encontrar un clustering adecuado?
	Si no lo logr\'o, explique por qu\'e el m\'etodo no puede encontrar tal clustering.
\end{pregunta}

\begin{pregunta}[extra]
	Describa un m\'etodo de clustering basado en persistencia en el que \texttt{RIPS\_SCALE} no deba estar necesariamente fijo.
	Podr\'ia este m\'etodo mejorar el resultado del algoritmo en el dataset 3?
\end{pregunta}

\section{Detecci\'on de periodicidad}


\section{Clasificaci\'on de poses}

\begin{pregunta}
	Explique la presencia de las dos clases m\'as persistentes en $H_1$ de la pose $8$.
	Considere usar un dibujo, pero sea preciso.
\end{pregunta}

\begin{pregunta}
	Explique por qu\'e las poses $1$ y $4$ son dif\'iciles de diferenciar usando s\'olo el diagrama de persistencia $H_0$.
	Explique por qu\'e el diagrama de persistencia $H_1$ si es efectivo para diferenciar las poses $1$ y $4$.
\end{pregunta}

\begin{pregunta}
	Describa el efecto de elegir un valor de \texttt{ALPHA} muy peque\~no.
	Qu\'e propiedad del diagrama de persistencia no est\'a siendo explotada al elegir un valor de \texttt{ALPHA} muy peque\~no?
\end{pregunta}

\end{document}